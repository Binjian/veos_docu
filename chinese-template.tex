\documentclass{article}
\usepackage{glossaries-cn}
\makeglossaries
%\usepackage[xindy]{imakeidx}
%\makeindex

%\loadglsentries[main]{INP-00-glossary}
\newacronym{ddye}{D$_{\text{dye}}$}{donor dye, ex. Alexa 488}
\newacronym[description={\glslink{r0}{F\"{o}rster distance}}]{R0}{$R_{0}$}{F\"{o}rster distance}
\newglossaryentry{r0}{name=\glslink{R0}{\ensuremath{R_{0}}},text=F\"{o}rster distance,description={F\"{o}rster distance, where 50\% ...}, sort=R}
\newglossaryentry{kdeac}{name=\glslink{R0}{\ensuremath{k_{DEAC}}},text=$k_{DEAC}$, description={is the rate of deactivation from ... and emission)}, sort=k} 
\begin{document}


%Term definitions
%\newglossaryentry{utc}{name=UTC, description={Coordinated Universal Time}}
%\newglossaryentry{adt}{name=ADT, description={Atlantic Daylight Time}}
%\newglossaryentry{est}{name=EST, description={Eastern Standard Time}}
%\newglossaryentry{computer}
%{
%    name = computer,
%    description={is a programmable machine}
%}


%

%Term definitions
\newglossaryentry{utc}{name=UTC, description={Coordinated Universal Time}}
\newglossaryentry{adt}{name=ADT, description={Atlantic Daylight Time}}
\newglossaryentry{est}{name=EST, description={Eastern Standard Time}}

%
%\newglossaryentry{computer}
%{
%  name=computer,
%  description={is a programmable machine that receives input,
%               stores and manipulates data, and provides
%               output in a useful format}
%}
%
%\newglossaryentry{naiive}
%{
%  name=na\"{\i}ve,
%  description={is a French loanword (adjective, form of naïf)
%               indicating having or showing a lack of experience,
%               understanding or sophistication}
%}

%\newglossaryentry{linux}
%{
%  name=Linux,
%  description={is a generic term referring to the family of Unix-like
%               computer operating systems that use the Linux kernel},
%  plural=Linuces
%}

%\newacronym[longplural={Frames per Second}]{fpsLabel}{FPS}{Frame per Second}
%\newglossaryentry{ex}{name={sample},description={an example}}

%\newacronym{svm}{SVM}{support vector machine}
%\longnewglossaryentry{computer}
%{
%  name=computer
%}
%  {is a programmable machine that receives input,
%               stores and manipulates data, and provides
%               output in a useful format}
%
%\newglossaryentry{pi}
%{
%  name={\ensuremath{\pi}},
%  description={ratio of circumference of circle to its
%               diameter},
%  sort=pi
%}
%


% \tableofcontents


%\begin{abstract}
%这是简介及摘要。
%\end{abstract}

\section{前言}

\section{关于数学部分}
数学、中英文皆可以混排.You can intersperse math, Chinese and English (Latin script) without adding extra environments.


%這是繁體中文。\gls{utc}
%
%
%Here's my \gls{est} term.
%
%
%
%\gls{adt}
%
%\gls{est}
%
\printglossaries

% \glsaddall
\end{document} 